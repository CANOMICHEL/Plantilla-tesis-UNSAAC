\chapter{Introducción}
\label{chap:cha1}

Este es el primer capítulo de la tesis. Se inicia con el desarrollo de la introducción de la tesis. En esta parte debes explicar que es lo que se presenta en el trabajo. ¿Por que se escogio este método?. Es importante que el texto utilice la tabla de abreviaturas correctamente. En el archivo abreviaturas.tex contiene la tabla de abreviaturas. Para citar alguna de ellas debes usar los comandos $\backslash$ac\{tu-sigla-aqui\}. Si es la primera vez que utilizas la sigla ella se expandirá por completo. Por ejemplo, el comando $\backslash$ac\{CMM\} va a producir: \ac{CMM}. Si más adelante repites el mismo comando sólo aparecerá la sigla \ac{CMM}. Para explorar mucho más este comando es necesario leer su manual disponible en: \href{http://www.ctan.org/tex-archive/macros/latex/contrib/acronym/}{Link referencia}.


\textbf{Hipótesis}.
Nuestra hipótesis consiste de la siguiente proposición: \textit{ Texto de la hipótesis}.


\section{El problema}
(En esta sección se realiza el planteamiento del problema que queremos resolver con la tesis. Sea muy puntual y no ocupe más de un párrafo en especificar cual es el problema que desea atacar)


\section{Justificación y motivación}
(En esta sección se va desde aspectos generales a aspectos específicos (como un embudo). No se olvide que es la primera parte que tiene contacto con el lector y que hará que este se interese en el tema a investigar).



\textbf{Justificación}.
A quien beneficia el trabajo de investigación una vez culminado, como se beneficia. y.
Especificar el problema que se va enfrentar 

\textbf{Motivación}.
¿Por qué es importante?, mensionar el aporte.


\section{Objectivo General}
En esta sección se colocan los objetivos generales de la tesis. Máximo dos. Si necesita ampliar estos objetivos utilice la sección de objetivos específicos.


\subsection{Objetivos Específicos}
En esta sección los objetivos especificos de la tesis son mencionados las cuales responderan las preguntas de la investigación.
 \begin{enumerate}
 	
	\item Primer objetivo.
	
	\item Segundo objetivo.
	
 	\item Tercer objetivo.
	
 \end{enumerate}

%\section{Contributions}

\section{Organización del texto}
Aquí deberas colocar como esta organizado tu texto.
