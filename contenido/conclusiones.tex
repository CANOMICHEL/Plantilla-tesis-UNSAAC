\chapter{Conclusiones y Trabajos Futuros}\label{chap:conclusions}

Las conclusiones de la tesis son una parte muy importante y tiene las siguientes partes.

En primer lugar debes escribir las conclusiones generales de tu trabajo, evita escribir en forma de viñetas. Simplemente utiliza texto continuo. En las concluciones es posible mencionar que es lo que se aprendio en el estudio de manera resumida. Realizar una conclucion no tienes que detallar como fue visto en los experimentos, las conclusiones son mejor apreciadas si son profundas.

\section{Limitaciones}
La segunda  parte de este capítulo corresponde a las limitaciones que tiene la propuesta. Esta seccion es muy importante para que los siguientes estudiantes que hagan algo en esta línea no cometan los mismos errores y tu tesis sea un buen peldaño para avanzar más rápido.

\section{Recomendaciones}
En esta sección el tesista debe reflejar que la tesis ha permitido adquirir nuevos conocimientos que podrían servir para guiar otros trabajos en el futuro.

\section{Trabajos futuros}
En base a los puntos anteriores es recomendable que tu tesis también sugiera trabajos futuros. Esta sección es esencialmente útil para otras ideas de tesis. Todo este capítulo no debe ser más de cuatro páginas.

