\chapter{Marco Teórico}
\label{chap:cha2}

Cada capítulo deberá contener una breve introducción que describe en forma rápida el contenido del mismo. En este capítulo va el marco teórico. (pueden ser dos capítulos de marco teórico)

Este capítulo presenta conceptos relacionados a nuestro método o abordaje de modo que el lector pueda familiarizarse con los temas expuestos.


Ejemplo de como escribir seudo código en latex.
\begin{algorithm}[!htb]
	\caption{\ac{AP}.}\label{alg:ap}
	\begin{algorithmic}[1]
		\Procedure{ClusteringAP}{$S$}
		\State $R(i,k) = 0, ~A(i,j)= 0, ~ \forall i,k$
		
		\While{ Until converge }
		\State $R(i,k) = S(i,k) - max(A(i,j) + S(i,j))~|~ (j \in [1,n]; j\neq k) $
		
		\State $A(i,k) = min (0, R(k,k) + \sum_j max(0, R(j,k))), ~|~ (j \in [1,n]; ~j\neq i; ~j\neq k )$
		
		\State $A(k,k) = \sum_i max(0,R(i,k)) , ~|~ (i \neq k)$ 
		
		\EndWhile
		\State \textbf{return} $Trks$
		\EndProcedure
	\end{algorithmic}
\end{algorithm} 

Ejemplo de como hacer definiciones en latex:

\begin{defn}
Given a set of $n + 1$ data points $(x_i, y_i)$ where no two $x_i$ are the same, there is a polynomial $p$ of degree at most $n$ with the property $p(x_i) = y_i$, $i=0,\ldots,n$.
\end{defn}

Ejemplo de ecuaciones: 
\begin{equation}
	\label{eqn:interpolation}
	 p(x) = a_n x^n + a_{n-1} x^{n-1} + \cdots + a_2 x^2 + a_1 x + a_0
\end{equation}

%\subsubsection{Sub sub sección} 
 
 Ejemplo de utilización de abreviaturas, \ac{SPC}.
 
 