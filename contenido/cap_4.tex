\chapter{Metodología}
\label{chap:cha4}

(En este capítulo se desarrolla toda la propuesta realizada a través de la investigación. Sigue la misma estructura del capítulo anterior. El título del capítulo es flexible de acuerdo a cada tesis. Aqui va una introduccion para el pipeline).

\section{Método}
Este capítulo explica el enfoque o solución para el problema.


\section{Abordaje}
En este capítulo puedes explicar detalladamente en que consiste tu solución.

Aqui podemos observar un ejemplo para escribir una definición:

\begin{defn}
Un punto $p$ es una tupla $(x, y, t)$, donde $x$ y $y$ son las posiciones en la imagen y $t$ es el lapso de tiempo cuando el punto es colectado, donde $k$ $\in$ $\mathds{N}$.
\begin{equation}
	p_k = (x_k,y_k,t_k).
\end{equation}
\end{defn}

%\noindent

Example of ecuación:
La ecuación \ref{equa:example}, es un ejemplo de como utilizar una ecuación.
\begin{equation}
    \label{equa:example}
	f(x_k) = y_k, k=1, ..., n
\end{equation}
